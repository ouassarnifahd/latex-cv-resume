\documentclass{faresume}
\usepackage[french]{babel}
\usepackage{CJKutf8}

\begin{document}

%%%%%%%%%%%%%%%%%%%%%%%%%%%%%%%%%%%%%%%%%%%%%%%%%%%%%%%%%%%%%%%%%%%%%
% HEADER
%%%%%%%%%%%%%%%%%%%%%%%%%%%%%%%%%%%%%%%%%%%%%%%%%%%%%%%%%%%%%%%%%%%%%
\profilepicture[{20cm 0cm 2cm 0cm}]{media/seri.jpg}

\author{Fahd Ouassarni}
\title{Informatique industrielle, Linux embarqué.}

\birth[FR]{17/09/1996}
\email{ouassarni@ecole.ensicaen.fr}
\address{6 Boulevard Mar\'echal Juin\\
         14000 CAEN}
\phone{+33 6 43 15 50 72}
\website{oussfahd.xyz}

\maketitle

%%%%%%%%%%%%%%%%%%%%%%%%%%%%%%%%%%%%%%%%%%%%%%%%%%%%%%%%%%%%%%%%%%%%%
% LEFT COLUMN
%%%%%%%%%%%%%%%%%%%%%%%%%%%%%%%%%%%%%%%%%%%%%%%%%%%%%%%%%%%%%%%%%%%%%
\begin{column}[\leftcolumnwidth]

    \addblock{\faRocket}{Workflow}

        \addcontent[free]
        {%
            \textmd{MPLABX} pour les projets PIC.
            \textmd{Vivado} pour les projets FPGA.\\
            \textmd{Eagle} et \textmd{LTSpice} pour les projets PCB.
            \textmd{VIM} et \textmd{GCC} pour les applications linux.\\
            \textmd{Yocto} ou \textmd{Buildroot} pour les projets linux embarqué.\\
            \underline{Basiquement une \textmd{toolchain} et un bon \textmd{éditeur de texte}}.
        }{}{}{}{}

    \addblock{\faBriefcase}{Expériences professionnelles}

        \addcontent[job]{Stagiaire (PFE)}
            {Hyptra}{Tailleville, France}
            {Avril 2019 --- Septembre 2019}
            {%
                \begin{additems}
                    \item Developpement de logiciel embarqué pour 8051.
                    \item Conception de circuit imprimé sur Altium Designer.
                    \item Synthèse FPGA avec PicoZed.
                \end{additems}
            }

        \addcontent[job]{Chef d'équipe technique}
            {SERI (ENSICAEN)}{Caen, France}
            {Septembre 2018}
            {%
                \begin{additems}
                    \item Start-up introduite par l'ENSICAEN.
                    \item Mission: Communication WIFI sur Linux Embarquée.
                    \item Produit: Space Inspection Rover [\link{https://www.facebook.com/watch/?v=335022830595514}{Video}]
                \end{additems}
            }

        \addcontent[job]{Étudiant chercheur}
            {Université de Kumamoto}{Kumamoto, Japon}
            {Mai 2018 --- Ao\^ut 2018}
            {%
                \begin{additems}
                    \item Laboratoire: Human Interface and Cyber Communication Laboratory.
                    \item Sujet de recherche: Multichannel speech segregation using FDBM on a Single Board Computer. [\link{https://github.com/ouassarnifahd/cfdbm}{cfdbm}]
                \end{additems}
            }

    \addblock{\faGraduationCap}{Diplômes et Formations}

        \addcontent[degree]{Diplôme d'ingénieur en électronique}
            {ENSICAEN}{Caen, France}
            {Septembre 2016 --- Mars 2020}
            {%
                \begin{additems}
                    \item Spécialité: Électronique et physique appliquée.
                    \item Majeur: Signal, automatique pour les télécommunication et l'embarqué.
                \end{additems}
            }

        \addcontent[degree]{Classes Préparatoires aux Grandes Écoles}
            {Lycée Jean Moulin}{Forbach, France}
            {Septembre 2014 --- Mai 2016}
            {%
                \begin{additems}
                    \item Spécialité: Mathématique, Physique et Science de l'ingénieur.
                    \item Résultat: ENSICAEN via Concours Commun Polytechnique.
                \end{additems}
            }

        \addcontent[degree]{Baccalauréat Scientifique}
            {Lycée Salman Al Farissi}{Salé, Maroc}
            {Septembre 2013 --- Juin 2014}
            {%
                \begin{additems}
                    \item Spécialité: Mathématique et Science de l'ingénieur.
                    \item Mention: Bien
                \end{additems}
            }

    \addblock{\faWrench}{Projets academiques}

        \addcontent[project]{OpenWRT on WiSoC}
            {}{Telecom, OpenWRT, Linux, MIPS}
            {Juin 2019 --- Présent}
            {%
                \begin{additems}
                    \item Introduction à différent concepts réseau en utilisant OpenWRT sur un mini-routeur WiFi.
                \end{additems}
            }

        \addcontent[project]{Xen on ARM}
            {}{Xen, Linux, FreeRTOS, ARM}
            {Mars 2019 --- Présent}
            {%
                \begin{additems}
                    \item Introduction à la virtualisation pour les systèmes embarqués avec Xen.
                \end{additems}
            }

        \addcontent[project]{SoC on FPGA}
            {ENSICAEN}{C, VHDL, Zynq, FPGA}
            {Decembre 2018 --- Février 2019}
            {%
                \begin{additems}
                    \item Introduction à l'architecture ARM/FPGA avec Zynq.
                \end{additems}
            }

        \addcontent[project]{Evaluation of TCI6638K2K}
            {THALES AIR SYSTEMS SAS}{SystemTap, Linux, ARM}
            {Octobre 2018 --- Février 2019}
            {%
                \begin{additems}
                    \item Introduction au outils de profiling sur un Linux embarquée pour des applications temps réel.
                \end{additems}
            }

        \addcontent[project]{Course Robot Mobile}
            {ENSICAEN}{LTSpice, Eagle, PCB}
            {Octobre 2017 --- Avril 2018}
            {%
                \begin{additems}
                    \item Réalisation d'un robot suiveur de ligne.
                \end{additems}
            }

\end{column}
%%%%%%%%%%%%%%%%%%%%%%%%%%%%%%%%%%%%%%%%%%%%%%%%%%%%%%%%%%%%%%%%%%%%%
% RIGHT COLUMN
%%%%%%%%%%%%%%%%%%%%%%%%%%%%%%%%%%%%%%%%%%%%%%%%%%%%%%%%%%%%%%%%%%%%%
\begin{column}[\rightcolumnwidth]

    \addblock{\faLightbulbO}{Motivation}

        \addcontent[free]
        {%
            {\Large\begin{CJK}{UTF8}{min}「学びは生涯の宝」\end{CJK}}
        }{}{}{}{}

    \addblock{\faDownload}{Atouts}

        \addcontent[asset]{Motivé}
        {}{}{}{}

        \addcontent[asset]{Curieux}
        {}{}{}{}

        \addcontent[asset]{Autonome}
        {}{}{}{}

        \addcontent[asset]{Déterminé}
        {}{}{}{}

    \addblock{\faComments}{Langues}

        \addcontent[lang]{Français}
            {0.85}
            {Bilingue}
            {}{}{}

        \addcontent[lang]{Anglais}
            {0.86}
            {TOEIC:~855/990}
            {}{}{}

        \addcontent[lang]{Arabe}
            {0.95}
            {Langue natale}
            {}{}{}

        \addcontent[lang]{Japonais}
            {0.28}
            {Débutant N4}
            {}{}{}

    \addblock{\faCogs}{Compétences techniques}

        \addcontent[asset]{Architectures}
            {x86, ARM, PIC, STM32, 8051.}
            {}{}{}{}

        \addcontent[asset]{Communication}
            {UART, SPI, I2C, CAN, USB.}
            {}{}{}{}

        \addcontent[asset]{Programmation}
            {Assembleur, C/C++, Java, Javascript.}
            {}{}{}{}

        \addcontent[asset]{Frameworks}
            {FreeRTOS, Yocto, CubeMX, NodeJS.}
            {}{}{}{}

        \addcontent[asset]{Simulation}
            {MATLAB, Simulink, LTSpice.}
            {}{}{}{}

        \addcontent[asset]{Scripting}
            {Shell, Python, Perl, Lua.}
            {}{}{}{}

        \addcontent[asset]{Markup}
            {HTML/CSS, {\LaTeX}, Markdown.}
            {}{}{}{}

        \addcontent[asset]{Réseau}
            {iptables, distcc, nfs, hostapd.}
            {}{}{}{}

        \addcontent[asset]{Outils}
            {Git, SSH, Make, Valgrind.}
            {}{}{}{}

        \addcontent[asset]{CAO}
            {Eagle, Altium Designer.}
            {}{}{}{}

        \addcontent[asset]{HDL}
            {VHDL, Verilog.}
            {}{}{}{}

    \addblock{\faFutbolO}{Centres d’intérêt}

        \addcontent[asset]{Technologie}
        {}{}{}{}

        \addcontent[asset]{Culture}
        {}{}{}{}

        \addcontent[asset]{Voyages}
        {}{}{}{}

        \addcontent[asset]{Jeux Vidéo}
        {}{}{}{}

    \addblock{\faGlobe}{Liens}

        \addcontent[asset]
        {\fawesomize{\faGithub}{}}
        {\link[\color{secondary-color}]{https://github.com/ouassarnifahd}{ouassarnifahd}}
        {inlined}{}{}

        \addcontent[asset]
        {\fawesomize{\faLinkedin}{}}
        {\link[\color{secondary-color}]{https://www.linkedin.com/in/fahd-ouassarni/}{fahd-ouassarni}}
        {inlined}{}{}

\end{column}

\end{document}
