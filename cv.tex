\documentclass{faresume}
\usepackage[french]{babel}
\usepackage{CJKutf8}

\begin{document}

%%%%%%%%%%%%%%%%%%%%%%%%%%%%%%%%%%%%%%%%%%%%%%%%%%%%%%%%%%%%%%%%%%%%%
% HEADER
%%%%%%%%%%%%%%%%%%%%%%%%%%%%%%%%%%%%%%%%%%%%%%%%%%%%%%%%%%%%%%%%%%%%%
\profilepicture[{20cm 0cm 2cm 0cm}]{seri.jpg}

\author{Fahd Ouassarni}
\title{Informatique industrielle, Linux embarqu\'e.}

\birth[FR]{17/09/1996}
\email{ouss.fahd.1996@gmail.com}
\address{6 Boulevard Mar\'echal Juin\\
		 14000 CAEN}
\phone{+33 6 43 15 50 72}
\website{oussfahd.xyz}

\maketitle

%%%%%%%%%%%%%%%%%%%%%%%%%%%%%%%%%%%%%%%%%%%%%%%%%%%%%%%%%%%%%%%%%%%%%
% LEFT COLUMN
%%%%%%%%%%%%%%%%%%%%%%%%%%%%%%%%%%%%%%%%%%%%%%%%%%%%%%%%%%%%%%%%%%%%%
\begin{column}[\leftcolumnwidth]

	\addblock{\faRocket}{Workflow}

		\addcontent
		{
			\textmd{MPLABX} pour les projets PIC.
			\textmd{Vivado} pour les projets FPGA.
			\textmd{Eagle} et \textmd{LTSpice} pour les projets PCB.
			\textmd{VIM} et \textmd{GCC} pour les projets Linux.\\
			\underline{Basiquement une \textmd{toolchain} et un bon \textmd{editeur de texte}}.
		}{}{}{}{}

	\addblock{\faBriefcase}{Exp\'eriences professionnelles}

		\addcontent[job]{Stagiaire}
			{Hyptra}{Tailleville, France}
			{Avril 2019 --- Septembre 2019}
			{
				\begin{additems}
					\item Developpement de logiciel embarqu\'e pour 8051.
					\item Conception de circuit imprim\'e sur Altium Designer.
					\item Synth\`ese FPGA avec PicoZed.
				\end{additems}
			}

		\addcontent[job]{Chef d'\'equipe technique}
			{SERI (ENSICAEN)}{Caen, France}
			{Septembre 2018}
			{
				\begin{additems}
					\item Jeux de r\^ole de startup introduit par l'ENSICAEN.
					\item Mission: Communication WIFI sur Linux Embarqu\'ee.
					\item Produit: Space Inspection Rover [\link{https://www.facebook.com/watch/?v=335022830595514}{Video}]
				\end{additems}
			}

		\addcontent[job]{\'Etudiant chercheur}
			{Universit\'e de Kumamoto}{Kumamoto, Japon}
			{Mai 2018 --- Ao\^ut 2018}
			{
				\begin{additems}
					\item Accueilli par: Human Interface and Cyber Communication Laboratory.
					\item Sujet de recherche: Multichannel speech segregation using FDBM on a Single Board Computer. [\link{https://github.com/ouassarnifahd/cfdbm}{cfdbm}]
				\end{additems}
			}

	\addblock{\faGraduationCap}{Dipl\^omes et Formations}

		\addcontent[degree]{Dipl\^ome d'ing\'enieur en \'electronique}
			{ENSICAEN}{Caen, France}
			{Septembre 2016 --- Septembre 2019}
			{
				\begin{additems}
					\item Sp\'ecialit\'e: \'Electronique et physique appliqu\'ee.
					\item Majeur: Signal, automatique pour les t\'el\'ecommunication et l'embarqu\'e.
				\end{additems}
			}

		\addcontent[degree]{Classes Pr\'eparatoires aux Grandes \'Ecoles}
			{Lyc\'ee Jean Moulin}{Forbach, France}
			{Septembre 2014 --- Mai 2016}
			{
				\begin{additems}
					\item Sp\'ecialit\'e: Math\'ematique, Physique et Science de l'ing\'enieur.
					\item R\'esultat: ENSICAEN via Concours Commun Polytechnique.
				\end{additems}
			}

		\addcontent[degree]{Baccalaur\'eat Scientifique}
			{Lyc\'ee Salman Al Farissi}{Sal\'e, Maroc}
			{Septembre 2013 --- Juin 2014}
			{
				\begin{additems}
					\item Sp\'ecialit\'e: Math\'ematique et Science de l'ing\'enieur.
					\item Mention: Bien
				\end{additems}
			}

	\addblock{\faWrench}{Projets professionnels}

		\addcontent[project]{Xen on ARM}
			{}{Xen, Linux, FreeRTOS, ARM}
			{Mars 2019}
			{
				\begin{additems}
					\item Introduction \`a la virtualisation pour les syst\`emes embarqu\'es avec Xen.
				\end{additems}
			}

		\addcontent[project]{SoC on FPGA}
			{ENSICAEN}{C, VHDL, Zynq, FPGA}
			{Decembre 2018 --- F\'evrier 2019}
			{
				\begin{additems}
					\item Introduction \`a l'architecture ARM/FPGA avec Zynq.
				\end{additems}
			}

		\addcontent[project]{Evaluation of TCI6638K2K}
			{THALES AIR SYSTEMS SAS}{Linux, C, ARM, DSP}
			{Octobre 2018 --- F\'evrier 2019}
			{
				\begin{additems}
					\item Introduction au Linux embarqu\'ee pour des applications temps r\'eel.
				\end{additems}
			}

		\addcontent[project]{cfdbm}
			{Universit\'e de Kumamoto}{C, ALSA, ARM, SBC}
			{Juin 2018 --- Ao\^ut 2018}
			{
				\begin{additems}
					\item Introduction \`a ALSA pour du traitement temps r\'eel du signal acoustique sur Linux.
				\end{additems}
			}

		\addcontent[project]{Course Robot Mobile}
			{ENSICAEN}{LTSpice, Eagle, PCB}
			{Octobre 2017 --- Avril 2018}
			{
				\begin{additems}
					\item R\'ealisation d'un robot suiveur de ligne.
				\end{additems}
			}

\end{column}
%%%%%%%%%%%%%%%%%%%%%%%%%%%%%%%%%%%%%%%%%%%%%%%%%%%%%%%%%%%%%%%%%%%%%
% RIGHT COLUMN
%%%%%%%%%%%%%%%%%%%%%%%%%%%%%%%%%%%%%%%%%%%%%%%%%%%%%%%%%%%%%%%%%%%%%
\begin{column}[\rightcolumnwidth]

	\addblock{\faLightbulbO}{Motivation}

		\addcontent
		{
			{\Large\begin{CJK}{UTF8}{min}「学びは生涯の宝」\end{CJK}}
		}{}{}{}{}

	\addblock{\faDownload}{Atouts}

		\addcontent[asset]{Motiv\'e}
		{}{}{}{}

		\addcontent[asset]{Curieux}
		{}{}{}{}

		\addcontent[asset]{Autonome}
		{}{}{}{}

		\addcontent[asset]{D\'etermin\'e}
		{}{}{}{}

		\addcontent[asset]{Travail collaboratif}
		{}{}{}{}

	\addblock{\faComments}{Langues}

		\addcontent[lang]{Arabe}
			{0.95}
			{Langue natale}
			{}{}{}

		\addcontent[lang]{Fran\c{c}ais}
			{0.85}
			{Bilingue}
			{}{}{}

		\addcontent[lang]{Anglais}
			{0.86}
			{TOEIC: 855/990}
			{}{}{}

		\addcontent[lang]{Japonais}
			{0.28}
			{D\'ebutant N4}
			{}{}{}

	\addblock{\faCogs}{Comp\'etences techniques}

		\addcontent[asset]{Architectures}
			{x86, ARM, PIC, 8051.}
			{}{}{}{}

		\addcontent[asset]{Communication}
			{UART, SPI, I2C, USB, WIFI, BLE, NFC.}
			{}{}{}{}

		\addcontent[asset]{Programmation}
			{C, C++, Java, Javascript.}
			{}{}{}{}

		\addcontent[asset]{Frameworks}
			{ALSA, SDL, OpenMP, Bootstrap.}
			{}{}{}{}

		\addcontent[asset]{Simulation}
			{MATLAB, Simulink.}
			{}{}{}{}

		\addcontent[asset]{Scripting}
			{Shell, Perl, Lua, Python, Matlab.}
			{}{}{}{}

		\addcontent[asset]{Markup}
			{HTML, CSS, {\LaTeX}, Markdown.}
			{}{}{}{}

		\addcontent[asset]{Outils}
			{Git, Makefile, FreeRTOS, Wireshark.}
			{}{}{}{}

		\addcontent[asset]{HDL}
			{VHDL, SPICE.}
			{}{}{}{}

	\addblock{\faFutbolO}{Centres d’int\'er\^et}

		\addcontent[asset]{Technologie}
		{}{}{}{}

		\addcontent[asset]{Culture}
		{}{}{}{}

		\addcontent[asset]{Voyages}
		{}{}{}{}

		\addcontent[asset]{Jeux Vid\'eo}
		{}{}{}{}

	\addblock{\faGlobe}{Liens}

		\addcontent[asset]
		{\fawesomize{\faGithub}{}}
		{\link[\color{secondary-color}]{https://github.com/ouassarnifahd}{ouassarnifahd}}
		{inlined}{}{}

		\addcontent[asset]
		{\fawesomize{\faLinkedin}{}}
		{\link[\color{secondary-color}]{https://www.linkedin.com/in/fahd-ouassarni/}{fahd-ouassarni}}
		{inlined}{}{}

\end{column}

\end{document}
