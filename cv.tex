\documentclass[a4paper]{article}

\usepackage[french]{babel}
\usepackage{CJKutf8}

%% packages used
\usepackage[hmargin=1.5cm,left=.4cm,vmargin=.4cm]{geometry}
\usepackage[hidelinks]{hyperref}
\usepackage[utf8]{inputenc}
\usepackage[default]{lato}
\usepackage[T1]{fontenc}
\usepackage{fontawesome}
\usepackage{graphicx}
\usepackage{setspace}
\usepackage{titlesec}
\usepackage{titling}
\usepackage{xcolor}
\usepackage{ifthen}
\usepackage{calc}

%% Font setting definitions
\newcommand{\authorfont}[0]{\huge\bfseries}
\newcommand{\titlefont}[0]{\large\bfseries}
\newcommand{\blocktitlefont}[0]{\large}
\newcommand{\sectiontitlefont}[0]{\normalsize\bfseries}
\newcommand{\contentfont}[0]{\normalsize\fontseries{l}\selectfont}
\newcommand{\tagsfont}[0]{\scriptsize\scshape}

%% DONT number pages!
\pagenumbering{gobble}

%% DONT use hyphenation
\tolerance=1
\hyphenpenalty=10000
\hbadness=10000

%% Color definitions
\definecolor{page-color}{HTML}{F6F2E3}
\definecolor{default-color}{HTML}{000000}
\definecolor{primary-color}{HTML}{ac7331}
\definecolor{secondary-color}{HTML}{848484}
\definecolor{dark-gray-color}{HTML}{444444}
\definecolor{light-gray-color}{HTML}{BFBFBF}

%% Nice taint
\pagecolor{page-color}

%% Length definitions
\newlength{\headerheigth}
\newlength{\headerwidth}
\newlength{\headersep}
\setlength{\headerheigth}           {2cm}
\setlength{\headerwidth}            {0.72\textwidth}
\setlength{\headersep}              {.5cm}

\newlength{\profilepicturewidth}
\newlength{\titlesectionwidth}
\setlength{\profilepicturewidth}    {0.20\headerwidth}
\setlength{\titlesectionwidth}      {0.60\headerwidth}

\newlength{\leftcolumnwidth}
\newlength{\rightcolumnwidth}
\newlength{\contactsectionwidth}
\setlength{\leftcolumnwidth}        {0.68\textwidth}
\setlength{\rightcolumnwidth}       {0.38\textwidth}
\setlength{\contactsectionwidth}    {0.90\rightcolumnwidth}

\newlength{\contentindent}
\newlength{\contentwidth}
\newlength{\contentsep}
\setlength{\contentindent}          {.5cm}
\newcommand{\contentwidthratio}[0]  {0.90}
\setlength{\contentsep}             {.2cm}

\newlength{\myrulefill}

%% For simplicity's sake!
\newcommand{\fawesomize}[3][t]{
    \makebox[\contentindent][c]{#2}
    \begin{minipage}[#1]{\contentwidth-\contentindent}
        #3
    \end{minipage}
    % \parbox[#1]{}
}

\newcommand{\squareitem}[2][t]{
    \fawesomize[#1]{{\tiny\faStop}}{#2}
}

% BUG invisible characters?
\newcommand{\link}[3][\color{primary-color}]{{#1\href{#2}{#3}}}

%% Header definitions
\newcommand{\profilepicture}[2][{0 0 0 0}]{
    \newcommand{\theprofilepicture}[0]{
    \begin{minipage}[c]{\profilepicturewidth}
        \includegraphics[trim=#1,clip,width=\textwidth]{#2}
    \end{minipage}
    \hspace{\contentindent}
    }
}
\newcommand{\birth}[2][EN]{
    \ifthenelse{\equal{#1}{FR}}
    {
        \newcommand{\thebirth}[0]{N\'e le #2}
    }
    {
        \newcommand{\thebirth}[0]{Date of birth #2}
    }
}
\newcommand{\email}[1]{\newcommand{\themail}[0]{
    \link[\color{secondary-color}]{mailto:#1}{#1}}
}
\newcommand{\address}[1]{\newcommand{\theaddress}[0]{#1}}
\newcommand{\phone}[1]{\newcommand{\thephone}[0]{#1}}
\newcommand{\website}[1]{\newcommand{\thewebsite}[0]{
    \link[\color{secondary-color}]{https://#1}{#1}}
}

%% Header Layout: [ Pic  Title		Contact]
\renewcommand{\maketitle}{
    \begin{minipage}[b][\headerheigth]{\headerwidth}
        \ifthenelse{\NOT\isundefined{\theprofilepicture}}
        {
            \theprofilepicture
        }
        {
            \setlength{\titlesectionwidth}{\leftcolumnwidth}
        }
    	\begin{minipage}[t][0cm][c]{\titlesectionwidth}
            {\color{default-color}\authorfont\theauthor}
            \par\setstretch{1.3}
            {\color{primary-color}\titlefont\thetitle}
    	\end{minipage}
    \end{minipage}\hfill
    \begin{minipage}[c][\headerheigth]{\contactsectionwidth}
        \color{secondary-color}\contentfont
        \setlength{\contentwidth}{\contentwidthratio\textwidth}
        \setstretch{1.2}
        \par\fawesomize{\faCalendar}{\thebirth}
        \par\fawesomize[c]{\faEnvelope}{\themail}
        \par\fawesomize{\faMapMarker}{\theaddress}\\
        \par\fawesomize{\faPhoneSquare}{\thephone}
        \par\fawesomize[c]{\faLink}{\thewebsite}
    \end{minipage}
    \par
    \ifthenelse{\isundefined{\theprofilepicture}}
    {
        \vspace{2\headersep}
    }
    {
        \vspace{\headersep}
    }
}

%% Body Layout:   [ COLUMN1         COLUMN2]
\newenvironment{column}[1][\textwidth]
{\begin{minipage}[t]{#1}}
{\end{minipage}\hfill}

%% Titles format and spacing definitions
% \titleformat{command}     % \titlespacing{command}
% [shape]                   % {left}
% {format}                  % {before-sep}
% {label}                   % {after-sep}
% {sep}                     % [right-sep]
% {before-code}
% [after-code]

%% section
\titleformat{\section}
{\color{primary-color}\blocktitlefont}
{}{0cm}{}
\titlespacing{\section}
{0cm}{\contentsep}{0.5\contentsep}

%% subsection
\titleformat{\subsection}[runin]
{\color{default-color}\sectiontitlefont}
{}{0cm}{}
\titlespacing{\subsection}
{0cm}{0cm}{3pt}[\contentsep]

%% New content
% Type [free, job, degree, project, lang, asset]
% ex: [free]:  {details}
%     [job,degree]:
%              {Position}{Company}{Place}{Period}{Details}
%     [project]:
%              {Project}[Company]{Tags}{Period}{Details}
%     [lang]:  {Skill}{Mastery}{Details}
%     [asset]: {Asset}{Details}
\newcommand{\addcontent}[6][free]{
    \hspace{\contentindent}
    \setlength{\contentwidth}{\contentwidthratio\textwidth}
    \begin{minipage}{\contentwidth}
        \ifthenelse{\NOT\equal{#1}{free}}
        {
            %% Position (Skill)
            \subsection{#2}
            \ifthenelse{\equal{#1}{asset}\AND
                        \NOT\equal{#3}{}\AND
                        \NOT\equal{#4}{inlined}\OR
                        \equal{#1}{lang}}
            {
                %% Breaking the runin subsection
                \hfill\\
            }
            {
                %% else put nothing in this paragraph
            }
        }
        {
            %% else put nothing in this paragraph
        }
        \ifthenelse{\equal{#1}{job}\OR\equal{#1}{degree}}
        {
            %% / Company Place
            %% Period
            {\contentfont / {\color{primary-color} #3} {\color{dark-gray-color}#4}}
            \par{\color{secondary-color}\contentfont #5}
            \par\vspace{0.6\contentsep}
        }
        {
            \ifthenelse{\equal{#1}{project}}
            {
                %% Tags
                \ifthenelse{\NOT\equal{#3}{}}
                {
                    {\contentfont / {\color{primary-color} #3}}
                }
                {
                    %% else put nothing in this paragraph
                }
                \hfill{\tagsfont #4}
                \par{\color{secondary-color}\contentfont#5}
                \par\vspace{0.6\contentsep}
            }
            {
                %% else put nothing in this paragraph
            }
        }
        \begin{minipage}{\textwidth}
            \ifthenelse{\equal{#1}{lang}}
            {
                %% Mastery (rule)
                \begin{minipage}{\contentwidthratio\contentwidth}
                    \setlength{\myrulefill}{\textwidth}
                    \addtolength{\myrulefill}{-#3\textwidth}
                    {\color{primary-color}\rule{#3\textwidth}{4pt}}
                    \hspace{-.13cm}
                    {\color{light-gray-color}\rule{\myrulefill}{4pt}}
                \end{minipage}
            }
            {
                %% else put nothing in this paragraph
            }
            %% Details (This paragraph is for everyone)
            \ifthenelse{\equal{#1}{free}}%
            {
                {\color{secondary-color}\contentfont{#2}}
            }
            {
                \ifthenelse{\equal{#1}{asset}}
                {
                    {\color{secondary-color}\contentfont{#3}}
                }
                {
                    \ifthenelse{\equal{#1}{lang}}
                    {
                        {\color{secondary-color}\contentfont{#4}}
                    }
                    {
                        \ifthenelse{\equal{#1}{degree}\OR
                                    \equal{#1}{job}\OR
                                    \equal{#1}{project}}
                        {
                            {\color{secondary-color}\contentfont{#6}}
                        }
                        {
                            %% else put nothing in this paragraph
                        }
                    }
                }
            }
        \end{minipage}
    \end{minipage}
    %% Spacing
    \ifthenelse{\equal{#1}{job}\OR\equal{#1}{degree}}%\OR\equal{#1}{project}}
    {
        \vspace{-0.5\contentsep}
    }
    {
        \vspace{0.5\contentsep}
    }
}

\newcommand{\addblock}[2]{
    \section{\fawesomize{#1}{\mbox{#2}}}
}


\begin{document}

%%%%%%%%%%%%%%%%%%%%%%%%%%%%%%%%%%%%%%%%%%%%%%%%%%%%%%%%%%%%%%%%%%%%%
% HEADER
%%%%%%%%%%%%%%%%%%%%%%%%%%%%%%%%%%%%%%%%%%%%%%%%%%%%%%%%%%%%%%%%%%%%%
\profilepicture[{20cm 0cm 2cm 0cm}]{seri.jpg}

\author{Fahd Ouassarni}
\title{Informatique industrielle, Linux embarqu\'e.}

\birth[FR]{17/09/1996}
\email{ouss.fahd.1996@gmail.com}
\address{6 Boulevard Mar\'echal Juin\\
		 14000 CAEN}
\phone{+33 6 43 15 50 72}
\website{oussfahd.xyz}

\maketitle

%%%%%%%%%%%%%%%%%%%%%%%%%%%%%%%%%%%%%%%%%%%%%%%%%%%%%%%%%%%%%%%%%%%%%
% LEFT COLUMN
%%%%%%%%%%%%%%%%%%%%%%%%%%%%%%%%%%%%%%%%%%%%%%%%%%%%%%%%%%%%%%%%%%%%%
\begin{column}[\leftcolumnwidth]

	\addblock{\faRocket}{Workflow}

		\addcontent
		{
			\textmd{MPLABX} pour les projets PIC.
			\textmd{Vivado} pour les projets FPGA.
			\textmd{Eagle} et \textmd{LTSpice} pour les projets PCB.
			\textmd{VIM} et \textmd{GCC} pour les projets Linux.\\
			\underline{Basiquement une \textmd{toolchain} et un bon \textmd{editeur de texte}}.
		}{}{}{}{}

	\addblock{\faBriefcase}{Exp\'eriences professionnelles}

		\addcontent[job]{Intern}
			{Hyptra}{Tailleville, France}
			{Avril 2019 --- Septembre 2019}
			{
				\squareitem{
					Mission: Developpement de logiciel embarqu\'e.
				}
			}

		\addcontent[job]{Chef d'\'equipe technique}
			{SERI (ENSICAEN)}{Caen, France}
			{Septembre 2018}
			{
				\squareitem{
					Jeux de r\^ole de startup introduit par l'ENSICAEN.
				}
				\squareitem{
					Mission: Communication WIFI sur Linux Embarqu\'ee.
				}
				\squareitem{
					Produit: Space Inspection Rover [\link{https://www.facebook.com/watch/?v=335022830595514}{Video}]
				}
			}

		\addcontent[job]{\'Etudiant chercheur}
			{Universit\'e de Kumamoto}{Kumamoto, Japan}
			{Mai 2018 --- Ao\^ut 2018}
			{
				\squareitem{
					Accueilli par: Human Interface and Cyber Communication Laboratory.
				}
				\squareitem{
					Sujet de recherche: Multichannel speech segregation using FDBM on a Single Board Computer. [\link{https://github.com/ouassarnifahd/cfdbm}{cfdbm}]
				}
			}

	\addblock{\faGraduationCap}{Dipl\^omes et Formations}

		\addcontent[degree]{Dipl\^ome d'ing\'enieur en \'electronique}
			{ENSICAEN}{Caen, France}
			{Septembre 2016 --- Septembre 2019}
			{
				\squareitem{
					Sp\'ecialit\'e: \'Electronique et physique appliqu\'ee.
				}
				\squareitem{
					Majeur: Signal, automatique pour les t\'el\'ecommunication et l'embarqu\'e.
				}
			}

		\addcontent[degree]{Classes Pr\'eparatoires aux Grandes \'Ecoles}
			{Lyc\'ee Jean Moulin}{Forbach, France}
			{Septembre 2014 --- Mai 2016}
			{
				\squareitem{
					Sp\'ecialit\'e: Math\'ematique, Physique et Science de l'ing\'enieur.
				}
				\squareitem{
					R\'esultat: ENSICAEN via Concours Commun Polytechnique.
				}
			}

		\addcontent[degree]{Baccalaur\'eat Scientifique}
			{Lyc\'ee Salman Al Farissi}{Sal\'e, Maroc}
			{Septembre 2013 --- Juin 2014}
			{
				\squareitem{
					Sp\'ecialit\'e: Math\'ematique et Science de l'ing\'enieur.
				}
				\squareitem{
					Mention: Bien
				}
			}

		\vspace{-.3cm} %% FIXME! Manual spacement

	\addblock{\faWrench}{Projets professionnels}

		\addcontent[project]{Xen on ARM}
			{}{ARM, Linux, Xen, Hypervisors}
			{Mars 2019}
			{
				\squareitem{
					Introduction \`a la virtualisation pour les syst\`emes embarqu\'ee avec Xen.
				}
			}

			\vspace{-.3cm} %% FIXME! Manual spacement

		\addcontent[project]{SoC on FPGA}
			{ENSICAEN}{C, VHDL, Zynq, FPGA}
			{Decembre 2018 --- F\'evrier 2019}
			{
				\squareitem{
					Introduction \`a l'architecture ARM/FPGA avec Zynq.
				}
			}

			\vspace{-.3cm} %% FIXME! Manual spacement

		\addcontent[project]{Evaluation of TCI6638K2K}
			{THALES AIR SYSTEMS SAS}{Linux, C, ARM, DSP}
			{Octobre 2018 --- F\'evrier 2019}
			{
				\squareitem{
					Introduction \`a Linux embarqu\'ee pour des applications temps r\'eel.
				}
			}

			\vspace{-.3cm} %% FIXME! Manual spacement

		\addcontent[project]{cfdbm}
			{Universit\'e de Kumamoto}{C, ALSA, ARM, SBC}
			{June 2018 --- Ao\^ut 2018}
			{
				\squareitem{
					Introduction \`a ALSA pour du traitement temps r\'eel du signal acoustique.
				}
			}

			\vspace{-.3cm} %% FIXME! Manual spacement

		\addcontent[project]{Course Robot Mobile}
			{ENSICAEN}{LTSpice, Eagle, PCB}
			{Octobre 2017 --- Avril 2018}
			{
				\squareitem{
					R\'ealisation d'un robot suiveur de ligne.
				}
			}

\end{column}
%%%%%%%%%%%%%%%%%%%%%%%%%%%%%%%%%%%%%%%%%%%%%%%%%%%%%%%%%%%%%%%%%%%%%
% RIGHT COLUMN
%%%%%%%%%%%%%%%%%%%%%%%%%%%%%%%%%%%%%%%%%%%%%%%%%%%%%%%%%%%%%%%%%%%%%
\begin{column}[\rightcolumnwidth]

	\addblock{\faLightbulbO}{Motivation}

		\addcontent
		{
			{\Large\begin{CJK}{UTF8}{min}「学びは生涯の宝」\end{CJK}}
		}{}{}{}{}

	\addblock{\faDownload}{Atouts}

		\addcontent[asset]{Motiv\'e}
		{}{}{}{}

		\addcontent[asset]{Curieux}
		{}{}{}{}

		\addcontent[asset]{Autonome}
		{}{}{}{}

		\addcontent[asset]{D\'etermin\'e}
		{}{}{}{}

		\addcontent[asset]{Travail collaboratif}
		{}{}{}{}

	\addblock{\faComments}{Langues}

		\addcontent[lang]{Arabe}
			{0.95}
			{Langue natale}
			{}{}{}

		\addcontent[lang]{Fran\c{c}ais}
			{0.85}
			{Bilingue}
			{}{}{}

		\addcontent[lang]{Anglais}
			{0.86}
			{TOEIC: 855/990}
			{}{}{}

		\addcontent[lang]{Japonais}
			{0.28}
			{D\'ebutant N4}
			{}{}{}

	\addblock{\faCogs}{Comp\'etences techniques}

		\addcontent[asset]{Architectures}
			{x86, ARM, PIC, 8051.}
			{}{}{}{}

		\addcontent[asset]{Communication}
			{UART, SPI, I2C, USB, WIFI, BLE, NFC.}
			{}{}{}{}

		\addcontent[asset]{Programmation}
			{C, C++, Java, Javascript.}
			{}{}{}{}

		\addcontent[asset]{Frameworks}
			{ALSA, SDL, OpenMP, Bootstrap.}
			{}{}{}{}

		\addcontent[asset]{Scripting}
			{Shell, Perl, Lua, Python, Matlab.}
			{}{}{}{}

		\addcontent[asset]{Markup}
			{HTML, CSS, {\LaTeX}, Markdown.}
			{}{}{}{}

		\addcontent[asset]{Outils}
			{Git, GCC, Makefile, SSH.}
			{}{}{}{}

		\addcontent[asset]{HDL}
			{VHDL, SPICE.}
			{}{}{}{}

	\addblock{\faFutbolO}{Centres d’int\'er\^et}

		\addcontent[asset]{Technologie}
		{}{}{}{}

		\addcontent[asset]{Culture}
		{}{}{}{}

		\addcontent[asset]{Voyages}
		{}{}{}{}

		\addcontent[asset]{Jeux Vid\'eo}
		{}{}{}{}

	\addblock{\faGlobe}{Liens}

		\addcontent[asset]
		{\fawesomize{\faGithub}{}}
		{\link[\color{secondary-color}]{https://github.com/ouassarnifahd}{ouassarnifahd}}
		{inlined}{}{}

		\addcontent[asset]
		{\fawesomize{\faLinkedin}{}}
		{\link[\color{secondary-color}]{https://www.linkedin.com/in/fahd-ouassarni/}{fahd-ouassarni}}
		{inlined}{}{}

\end{column}

\end{document}
